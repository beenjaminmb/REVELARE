\documentclass{article}
\usepackage{mathtools}
\usepackage{fullpage}
\usepackage{cals}
\usepackage{listings}
\usepackage{tabularx}
\begin{document}
{\bf Definitions:}\\
Taint of a register is defined as $\overline{reg}$\\
Similarly, taint of a memory location pointed to by the register is defined\\
as $\overline{[reg]}$ \\
Taint flows from $b$ to $a$ is defined as $a\leftarrow{b}$
\\\\
I'm wondering if we should always end in a copy dependency. This seems
like the correct thing to do,as assignment is always the last operation
to be performed semantically.

Before we had the following for the example (from x86) mov edx, [ecx]:
\begin{align*}
        tmp_1 \leftarrow copy\_dep(\overline{[ecx]})\\
        \overline{edx} \leftarrow load\_address\_dep(data\_taint=tmp_1, address\_taint=\overline{ecx})
\end{align*}

I think we should instead think of it as:
\begin{align*}
        tmp_1 \leftarrow load\_address\_dep(data\_taint=\overline{[ecx]}, address\_taint=\overline{ecx})\\
        \overline{edx} \leftarrow copy\_dep(tmp_1)
\end{align*}

More complex load address dependency:
Our example is from x86. mov edx, [edx+esi*2]
\begin{align*}
        tmp_1 \leftarrow copy\_dep(\overline{[edx+esi*2]})\\
tmp_2 \leftarrow computation\_dep(\overline{edx},\overline{esi}) \\
        \overline{edx} \leftarrow load\_address\_dep(address\_taint=tmp_1,data\_taint=tmp_2)
\end{align*}
This would change to:
\begin{align*}
tmp_1 \leftarrow computation\_dep(\overline{edx},\overline{esi}) \\
        tmp_2 \leftarrow load\_address\_dep(data\_taint=\overline{[edx+esi*2]},address\_taint=tmp_1)\\
        \overline{edx} \leftarrow copy\_dep(tmp_2)
\end{align*}
For store address dependencies:
Our example is from x86. mov [ebx], esi
\begin{align*}
        tmp_1 \leftarrow copy\_dep(\overline{esi})\\
        \overline{[ebx]} \leftarrow store\_address\_dep(address\_taint=\overline{ebx},data\_taint=tmp_1)
\end{align*}
With the new way of thinking:
\begin{align*}
        tmp_1 \leftarrow store\_address\_dep(address\_taint=\overline{ebx},data\_taint=\overline{esi})\\
        \overline{[ebx]} \leftarrow copy\_dep(\overline{tmp_1})
\end{align*}
For store address dependencies:
Our example is from x86. mov [ebx+ecx], esi
\begin{align*}
        tmp_1 \leftarrow computation\_dep(\overline{ebx},\overline{ecx})\\
        tmp_2 \leftarrow copy\_dep(\overline{esi})\\
        \overline{[ebx+ecx]} \leftarrow store\_address\_dep(address\_taint=tmp_1,data\_taint=tmp_2)
\end{align*}
With the new way of thinking:
\begin{align*}
        tmp_1 \leftarrow computation\_dep(\overline{ebx},\overline{ecx})\\
        tmp_2 \leftarrow store\_address\_dep(address\_taint=tmp_1,data\_taint=\overline{esi})\\
        \overline{[ebx+ecx]} \leftarrow copy\_dep(\overline{tmp_2})
\end{align*}
It is possible to have both a load and store in some architectures,
such as x86. For example: push [ecx - 0x4] achieves;
\begin{verbatim}
esp -= 4 
[esp] = ([ecx - 0x4])
\end{verbatim}
. The second line from above has many dependencies.
\begin{align*}
        tmp_1 \leftarrow copy\_dep(\overline{[ecx-0x4]})\\
        tmp_2 \leftarrow load\_address\_dep({tmp_1,\overline{ecx}})\\
        \overline{[esp]} \leftarrow store\_address\_dep(tmp_2,\overline{esp})
\end{align*}
This new way of thinking explains why we have only one copy dependency
even though both load and store address dependencies usually have one copy each,
we only have one here, because there is just one assignment.
\begin{align*}
        tmp_1 \leftarrow load\_address\_dep(data\_taint =\overline{ecx},address\_taint=\overline{[ecx - 0x4]})\\
        tmp_2 \leftarrow store\_address\_dep(address\_taint=\overline{esp},data\_taint=tmp_1])\\
        \overline{[esp]} \leftarrow copy\_dep(\overline{tmp2})
\end{align*}

Briefly the difference between Suh et al and Espinoza et al:
Suh et al, used a bit (0,1) to mark if something was "spurious" or not,
which is the same as tainted or clean.
Espinoza et al used vectors to convey taint information, therefore a taint
mark is a vector of size $n$.This allows the taint system to convey
a rough estimate of its source, a cosine similarity can show how
similar two taint marks in the system are. This allows us to measure
"how much" taint a mark has, as well as where a mark is from.
\begin{table}
\centering
        \begin{tabular}{l | l}
                \begin{tabular}{l l}
                        \multicolumn{2}{c}{Suh et al}\\ \hline
                        (1) & $\overline{a} \leftarrow \overline{b} |
                         \overline{a}$\\
                        \multicolumn{2}{c}{Where $|$ is bitwise or.}\\
                   
                \end{tabular}
        &
                \begin{tabular}{l l}
                        \multicolumn{2}{c}{Espinoza et al}\\\hline
                        (1) & $\overline{a} \leftarrow \overline{a} + \overline{b}$\\
                        \multicolumn{2}{c}{Where $+$ is vector addition}
                \end{tabular}
                
                
        \end{tabular}
        \caption{Copy Dependency}
\end{table}

\begin{table}
\centering
        \begin{tabular}{l | l}
                \begin{tabular}{l l}
                        \multicolumn{2}{c}{Suh et al}\\ \hline
                        (1) & $\overline{a} \leftarrow \overline{b} |
                         \overline{a}$\\
                        \multicolumn{2}{c}{Where $|$ is bitwise or.}\\
                   
                \end{tabular}
        &
                \begin{tabular}{l l}
                        \multicolumn{2}{c}{Espinoza et al}\\\hline
                        (1) & $\overline{a} \leftarrow \overline{a} + \overline{b}$\\
                        \multicolumn{2}{c}{Where $+$ is vector addition}
                \end{tabular}
                
                
        \end{tabular}
        \caption{Computation Dependency}
\end{table}
\begin{table}
\centering
        \begin{tabular}{l | l}
                \begin{tabular}{l l}
                        \multicolumn{2}{c}{Suh et al}\\ \hline
                        (1) & $\overline{a} \leftarrow \overline{b} |
                         \overline{a}$\\
                        \\
                        \multicolumn{2}{c}{Where $|$ is bitwise or.}\\
                   
                \end{tabular}
        &
                \begin{tabular}{l l}
                        \multicolumn{2}{c}{Espinoza et al}\\\hline
                        (1) & $\overline{a} \leftarrow \overline{a} + \overline{b}$\\
                        (2) & $\overline{a} = \overline{a} * \beta$\\
                        \multicolumn{2}{c}{Where $+$ is vector addition and
                        $\beta$ is a value used to scale a}
                \end{tabular}
                
                
        \end{tabular}
        \caption{Load Address Dependency}
\end{table}
\begin{table}
\centering
        \begin{tabular}{l | l}
                \begin{tabular}{l l}
                        \multicolumn{2}{c}{Suh et al}\\ \hline
                        (1) & $\overline{a} \leftarrow \overline{b} |
                         \overline{a}$\\
                        \\
                        \multicolumn{2}{c}{Where $|$ is bitwise or.}\\
                   
                \end{tabular}
        &
                \begin{tabular}{l l}
                        \multicolumn{2}{c}{Espinoza et al}\\\hline
                        (1) & $\overline{a} \leftarrow \overline{a} + \overline{b}$\\
                        (2) & $\overline{a} = \overline{a} * \beta$\\
                        \multicolumn{2}{c}{Where $+$ is vector addition and
                        $\beta$ is a value used to scale a}
                \end{tabular}
                
                
        \end{tabular}
        \caption{Store Address Dependency}
\end{table}
\end{document}
